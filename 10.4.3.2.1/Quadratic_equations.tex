\title{linear equations in two variables}
%\usepackage{graphicx} % Required for inserting images

\documentclass[12pt]{article}
\usepackage{amsmath}
\newcommand{\myvec}[1]{\ensuremath{\begin{pmatrix}#1\end{pmatrix}}}
\newcommand{\mydet}[1]{\ensuremath{\begin{vmatrix}#1\end{vmatrix}}}
\newcommand{\solution}{\noindent \textbf{Solution: }}
\providecommand{\brak}[1]{\ensuremath{\left(#1\right)}}
%\providecommand{\norm}[1]{\left\lVert#1\right\rVert}
\let\vec\mathbf

\title{Quadratic Equations}
\author{G.sai avinash (gsavinash@sriprakashschools.com)}

\begin{document}
\maketitle
\section*{Class 10$^{th}$ Maths - Chapter 4}
This is Problem-2 from Exercise 4.3
\begin{enumerate}
\item   Find the roots of the quadratic equations by applying the quadratic formula \\

${(i)2{x^2} - 7x + 3 = 0}$


\solution \\
Given Data:

${2x^2- 7x + 3 = 0}$\\
Quadratic formula\\

\begin{aligin}
$x=\frac{-b\pm\sqrt{b^2-4ac}}{2a}$\\
$x=\frac{-(-7)\pm\sqrt{(-7)^2-4x2x3 }}{2 \times 2}$\\
$x=\frac{7\pm\sqrt{49 - 24}}{4}$\\
$x=\frac{7\pm\sqrt{25}}{4}$\\
\end{aligin}


1st 

\begin{aligin}
$x=\frac{7+5}{4}$\\
$x=\frac{12}{4}$\\
$x=3$\\
\end{aligin}


2nd condition

\begin{aligin}
$x=\frac{7-5}{4}$\\
$x=\frac{2}{4}$\\
$x=\frac{1}{2}$\\
\end{aligin}



hence the roots are: ${x=\frac{1}{2}}{, x=3}$\\









\end{enumerate}



\end{document}