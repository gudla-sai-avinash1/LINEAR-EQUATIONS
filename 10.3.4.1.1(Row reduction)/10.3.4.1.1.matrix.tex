\documentclass[12pt]{article}
\usepackage{amsmath}
\newcommand{\myvec}[1]{\ensuremath{\begin{pmatrix}#1\end{pmatrix}}}
\newcommand{\mydet}[1]{\ensuremath{\begin{vmatrix}#1\end{vmatrix}}}
\newcommand{\solution}{\noindent \textbf{Solution: }}
\providecommand{\brak}[1]{\ensuremath{\left(#1\right)}}
%\providecommand{\norm}[1]{\left\lVert#1\right\rVert}
\let\vec\mathbf

\title{Pair of linear equation in two variables}
\author{G.sai avinash (gsavinash@sriprakashschools.com)}

\begin{document}
\maketitle
\section*{Class 10$^{th}$ Maths - Chapter 3}
This is Problem-1 from Exercise 3.4
\begin{enumerate}
\item   Solve the following pair of linear equations by elimination method and substitution method:                         
\begin{align}
    x+y=5\\
        2x-3y=4
\end{align}
\solution\\
The equations can be written as:\\
\begin{align}
\myvec{1&1&5\\2&-3&4}
\end{align}
$R_1 \xrightarrow\ 3R_1 + R_2$\\ 
we get,
\begin{align}
\myvec{5&0&19\\2&-3&4}
\end{align}
$R_1 \xrightarrow\ \frac{R_1}{5}$\\ 
\begin{align}
\myvec{1&0&\frac{19}{5}\\2&-3&4}
\end{align}
$R_2 \xrightarrow\ R_2 - 2R_1$\\
\begin{align}
\myvec{1&0&\frac{19}{5}\\0&-3&\frac{-18}{5}}
\end{align}
$R_2 \xrightarrow\ \frac{R_2}{-3}$\\
\begin{align}
\myvec{1&0&\frac{19}{5}\\0&1&\frac{6}{5}}
\end{align}
Therefore,
\begin{align}
  x &= \frac{19}{5}\\
  y &= \frac{6}{5}
\end{align}


\end{enumerate}
\end{document}
