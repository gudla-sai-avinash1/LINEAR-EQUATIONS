\title{linear equations in two variables}
%\usepackage{graphicx} % Required for inserting images

\documentclass[12pt]{article}
\usepackage{amsmath}
\newcommand{\myvec}[1]{\ensuremath{\begin{pmatrix}#1\end{pmatrix}}}
\newcommand{\mydet}[1]{\ensuremath{\begin{vmatrix}#1\end{vmatrix}}}
\newcommand{\solution}{\noindent \textbf{Solution: }}
\providecommand{\brak}[1]{\ensuremath{\left(#1\right)}}
%\providecommand{\norm}[1]{\left\lVert#1\right\rVert}
\let\vec\mathbf

\title{Coordinate Geometry}
\author{G.sai avinash (gsavinash@sriprakashschools.com)}

\begin{document}
\maketitle
\section*{Class 10$^{th}$ Maths - Chapter 7}
This is Problem-7 from Exercise 7.4
\begin{enumerate}
\item   Let A(4, 2), B(6,5) and C(1, 4) be the vertices of triangle ABC\\
(ii)Find the coordinates of the point P on the AD, such that \\
AP: PD = 2: 1.\\
\solution:\\
Median AD of the triangle will divide the side BC in two equal parts. So D is the midpoint of side BC\\ 
\begin{align}
Coordinates of D =\myvec{ \frac{(6+1)}{2} &
+ \frac{(5+4)}{2}} = &
\myvec{ \frac{7}{2} &
+ \frac{9}{2}}
\end{align}

Point P divides the side AD in a ratio 2:1.\\
\begin{align}
Coordinates of P =\frac{\frac{2 x }&\frac{7}{2}&\frac{1 x 1}}{2+1}\\

\end{align}



\end{enumerate}



\end{document}