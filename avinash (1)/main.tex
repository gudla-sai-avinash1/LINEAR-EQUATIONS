

\title{linear equations in two variables}
%\usepackage{graphicx} % Required for inserting images

\documentclass[12pt]{article}
\usepackage{amsmath}
\newcommand{\myvec}[1]{\ensuremath{\begin{pmatrix}#1\end{pmatrix}}}
\newcommand{\mydet}[1]{\ensuremath{\begin{vmatrix}#1\end{vmatrix}}}
\newcommand{\solution}{\noindent \textbf{Solution: }}
\providecommand{\brak}[1]{\ensuremath{\left(#1\right)}}
%\providecommand{\norm}[1]{\left\lVert#1\right\rVert}
\let\vec\mathbf

\title{Pair of linear equation in two variables}
\author{G.sai avinash (gsavinash@sriprakashschools.com)}

\begin{document}
\maketitle
\section*{Class 10$^{th}$ Maths - Chapter 3}
This is Problem-1 from Exercise 3.4
\begin{enumerate}
\item   Solve the following pair of linear equations by elimination method and substitution method:                         
\begin{align}
    x+y=5\\
        2x-3y=4
\end{align}
\solution\\
Equation can be written as:\\
\begin{align}
\myvec{1&2\\ 2&-3}\myvec{x\\y} = \myvec{5\\4}\\
x=\frac{{\mydet{ \vec{b} & \vec{a_2}}}}{\mydet{\vec{a_1} & 
\vec{a_2}}}=&\frac{\mydet{5 & 4 \\ 1 & -3}}{\mydet{1 &2 \\ 1 & -3}}=
\frac{(5)(-3) - (1)(4)}{(1)(-3) - (1)(2)}=
\frac{-15 -4}{-3-2}=
\frac{-19}{-5}=
\frac{19}{5}\\
y=\frac{{\mydet{ \vec{a_1} & \vec{b}}}}{\mydet{\vec{a_1} & 
\vec{a_2}}}=&\frac{\mydet{1 & 2 \\ 5 & 4}}{\mydet{1 &2 \\ 1 & -3}}=
\frac{(1)(4) - (5)(2)}{(1)(-3) - (1)(2)}=
\frac{4 -10}{-3-2}=
\frac{-6 }{-5}=
\frac{6}{5}
\end{align}
 x and y are consistent and have unique solution.










\end{enumerate}



\end{document}
